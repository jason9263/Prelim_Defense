%%%% my notation
\parindent = 5ex

\newcommand\ams{Department of Applied Mathematics and Statistics}
\newcommand\bnl{Center for Data Intensive Computing\\Brookhaven National Laboratory}

\newcommand{\half}{\mathchoice
 {\frac{1}{2}} {1/2} {\frac{1}{2}} {1/2}}

\newcommand{\quarter}{\mathchoice
 {\frac{1}{4}} {1/4} {\frac{1}{4}} {1/4}}
                             
\newcommand{\ud}{\mathrm{d}}

%%%% end of my notation

%
\newcommand{\doublespacing}{\renewcommand{\baselinestretch}{1.66}\normalsize}
\newcommand{\singlespacing}{\renewcommand{\baselinestretch}{1.00}\normalsize}
\newcommand{\beq}{\begin{equation}}
\newcommand{\eeq}{\end{equation}}
\newcommand{\bea}{\begin{eqnarray}}
\newcommand{\eea}{\end{eqnarray}}
%
% TODO enviroment
\newcommand{\todo}[1]{[*** TO DO: #1 ***]}
\newenvironment{TODO}
 {\goodbreak\medskip\par\noindent
   TO DO:\par\noindent$\overline{*****}$\begin{itemize}}
    {\end{itemize}\par\noindent$\underline{*****}$\medskip}
%
% Macros for noninteracting cluster theory and vapor condensation:
% 
\newcommand{\dissty}{\displaystyle}
\newcommand{\scrsty}{\scriptstyle}
%
% cross-referencing
%\newcommand{\eqref}[1]{(\ref{#1})}
\newcommand{\FronTierp}{\textit{\hbox{Fron\hspace{-2pt}T\hspace{-1pt}ier++\hspace{4pt}}}}
\newcommand{\Equation}[1]{Equation~\eqref{#1}}
\newcommand{\Equations}[1]{Equations~\eqref{#1}}
\newcommand{\Eqn}[1]{Eq.~\eqref{eqn:#1}}
\newcommand{\Eqs}[1]{Eqs.~\eqref{#1}}
\newcommand{\Eqsthru}[2]{Eqs.~\eqref{#1}--\eqref{#2}}
\newcommand{\Eqand}[2]{Eqs.~\eqref{#1} and~\eqref{#2}}
\newcommand{\Equationsand}[2]{Equations~\eqref{#1} and~\eqref{#2}}
\newcommand{\Eqor}[2]{Eq.~\eqref{#1} or~\eqref{#2}}
\newcommand{\Fig}[1]{Fig.~\ref{fig:#1}}
\newcommand{\Figure}[1]{Figure.~\ref{#1}}
\newcommand{\Figs}[2]{Figs.~\ref{#1}--\ref{#2}}
\newcommand{\Sec}[1]{Sec.~\ref{Sec:#1}}
\newcommand{\Secs}[2]{Secs.~\ref{#1}--\ref{#2}}
\newcommand{\Chap}[1]{Chapter~\ref{Chap:#1}}
\newcommand{\Tab}[1]{Table~\ref{tab:#1}}
\newcommand{\old}[1]{
{}}

%
\newcommand{\vaporliquid}{vapor-liquid{ }}
%
% units
\newcommand{\degC}{\mbox{$^\circ$C}{}}
\newcommand{\degF}{\mbox{$^\circ$F}{}}
\newcommand{\bars}{\mbox{bars}{}}
\newcommand{\meters}{\mbox{m}{}}
\newcommand{\grams}{\mbox{g}{}}
\newcommand{\mm}{\mbox{mm}{}}
\newcommand{\cm}{\mbox{cm}{}}
\newcommand{\nm}{\mbox{nm}{}}
%\newcommand{\um}{\mbox{$\mu$m}{}}
\newcommand{\secs}{\mbox{s}{}}
\newcommand{\msec}{\mbox{ms}{}}
\newcommand{\usec}{\mbox{$\mu$s}{}}
\newcommand{\nsec}{\mbox{ns}{}}
\newcommand{\cmpers}{\mbox{cm/s}{}}
\newcommand{\mpers}{\mbox{m/s}{}}
\newcommand{\ergperg}{\mbox{erg/g}{}}
\newcommand{\ccperg}{\mbox{cm$^3$/g}{}}
\newcommand{\gpercc}{\mbox{g/cm$^3$}{}}
\newcommand{\gpercmsq}{\mbox{g/cm$^2$}{}}
\newcommand{\ergpergK}{\mbox{erg/g-K}{}}
\newcommand{\nperccpers}{\mbox{nuclei/cm$^3$-s}{}}
\newcommand{\percc}{\mbox{/cm$^3$}{}}
\newcommand{\perccpers}{\mbox{cm$^{-3}$s$^{-1}$}{}}
%
% numeral adjectives
\newcommand{\St}{{\mbox{\small st}}}
\newcommand{\Nd}{{\mbox{\small nd}}}
\newcommand{\Rd}{{\mbox{\small rd}}}
\newcommand{\Th}{{\mbox{\small th}}}
%
% Latin abbreviations
\newcommand{\etc}{{\em etc}}
\newcommand{\ie}{{\em i.e., }}
\newcommand{\Ie}{{\em I.e., }}
\newcommand{\cf}{{\em cf.~}}
\newcommand{\etal}{{\em et~al.}}
\newcommand{\eg}{{\em e.g., }}
\newcommand{\viz}{{\em viz., }}
\newcommand{\vs}{{\em vs.~}}
\newcommand{\via}{{\em via }}
\newcommand{\viceversa}{{\em vice-versa}}
%
% For the following, math mode is assumed.
%
% Regular derivatives
\newcommand{\oderiv}[2]{{\displaystyle \frac{d #1}{d #2}}}
\newcommand{\pderiv}[2]{{\displaystyle \frac{\partial {#1}}{\partial {#2}}}}
\newcommand{\pderivc}[3]{{\displaystyle \left(\pderiv{#1}{#2}\right)_{#3}}}
%
% Thermodynamic derivatives
\newcommand{\thermoDeriv}[3]{{\displaystyle \left(\frac{\partial {#1}}{\partial {#2}}\right)_{#3}}}
\newcommand{\thermoDerivInline}[3]{{\displaystyle \left(\partial {#1}/\partial {#2}\right)_{#3}}}
\newcommand{\thermoTwoDeriv}[3]{{\displaystyle \left(\frac{\partial^2 {#1}}{\partial {#2}^2}\right)_{#3}}}
\newcommand{\thermoTwoDerivInline}[3]{{\displaystyle \left(\partial^2
{#1}/\partial {#2}^2\right)_{#3}}}
%
% Ways to display fractions with the proper size
\newcommand{\sfrac}[2]{{\textstyle \frac{#1}{#2}}}
\newcommand{\bfrac}[2]{{\displaystyle \frac{#1}{#2}}}
%
%
%%%%%%%%%%%%%%%%%%%%%%%%%%%%  Notation %%%%%%%%%%%%%%%%%%%%%%%%%%%%%%
%
\newcommand{\D}{\mathcal{D}}
\newcommand{\Dtilde}{\tilde{\mathcal{D}}}
% Riemann problem notation 
\newcommand{\Jfr}{J_{\mbox{\scriptsize fr}}}
\newcommand{\Jss}{J_{\mbox{\scriptsize ss}}}
\newcommand{\Mfr}{M_{\mbox{\scriptsize fr}}}
\newcommand{\Mss}{M_{\mbox{\scriptsize ss}}}
\newcommand{\Ifr}{I_{\mbox{\scriptsize fr}}}
\newcommand{\Iss}{I_{\mbox{\scriptsize ss}}}
\newcommand{\ufr}{u_{\mbox{\scriptsize fr}}}
\newcommand{\uss}{u_{\mbox{\scriptsize ss}}}
\newcommand{\Pfr}{P_{\mbox{\scriptsize fr}}}
\newcommand{\Pss}{P_{\mbox{\scriptsize ss}}}
\newcommand{\Dfr}{D_{\mbox{\scriptsize fr}}}
\newcommand{\Dss}{D_{\mbox{\scriptsize ss}}}
%
% Important cluster and droplet sizes and rates
\newcommand{\ivapor}{i_v}
\newcommand{\imax}{i_{\max}}
\newcommand{\icrit}{i_*}
\newcommand{\mcrit}{m_*}
\newcommand{\rcrit}{r_*}
\newcommand{\idrop}{i_o}
\newcommand{\mdrop}{m_o}
\newcommand{\rdrop}{r_o}
\newcommand{\dsize}{i_d}
\newcommand{\dmass}{m_d}
\newcommand{\dradius}{r_d}
\newcommand{\dtemp}{T_d}
\newcommand{\igrowth}{\Omega_i}
\newcommand{\mgrowth}{\Omega_m}
\newcommand{\rgrowth}{\Omega_r}
\newcommand{\ratevars}{{\bf \Lambda}}
\newcommand{\ratecoeffs}{{\bf R}}
\newcommand{\corrfac}{\Gamma}
%
% Reduced and dimensionless variables
\newcommand{\Phat}{\widehat{P}}
\newcommand{\Vhat}{\widehat{V}}
\newcommand{\That}{\widehat{T}}
\newcommand{\Stilde}{\widetilde{S}}
\newcommand{\Etilde}{\widetilde{E}}
\newcommand{\CVtilde}{\widetilde{C}_V}
\newcommand{\Btilde}[1]{\widetilde{B}_{#1}}
%
\newcommand{\Tref}{T_{\mbox{\scriptsize ref}}}
\newcommand{\Trefscr}{T_{\mbox{\scriptsize ref}}}
\newcommand{\Reff}{R_{\mbox{\scriptsize eff}}}
\newcommand{\State}{{\bf U}}
\newcommand{\state}{U}
\newcommand{\Flux}{{\bf f}}
\newcommand{\Numflux}{{\bf F}}
\newcommand{\massflux}{{\cal M}}
\newcommand{\entropy}{S}
\newcommand{\sat}{{\cal S}}
\newcommand{\Sonebar}{\overline{\cal S}_1}
\newcommand{\Sonehat}{\widehat{\cal S}_1}
\newcommand{\Hug}{{\cal H}}
\newcommand{\Kn}{\mbox{\em Kn}}
\newcommand{\smallKn}{\mbox{\small\em Kn}}
\newcommand{\QN}[2]{\frac{Q_{#1}^{#2}}{{#2}!}}
\newcommand{\sumto}[2]{\sum_{{#1}=1}^{#2}}
\newcommand{\sumtoinf}[1]{\sum_{{#1}=1}^{\infty}}
\newcommand{\fprime}[1]{f_{#1}^\prime}
\newcommand{\Nbar}[1]{\overline{N}_{#1}}
\newcommand{\nbar}[1]{\overline{n}_{#1}}
\newcommand{\mubar}[1]{\overline{\mu}_{#1}}
\newcommand{\Gbar}{\overline{G}}
\newcommand{\Cbar}[1]{\overline{C}_{#1}}
\newcommand{\ubar}{\overline{u}}
\newcommand{\muo}[1]{\mu_{#1}^{o}}
\newcommand{\muos}[1]{\mu_{#1,s}^{o}}
\newcommand{\nis}[1]{n_{#1,s}}
\newcommand{\licubed}{\lambda_{i}^{3}}
%
\newcommand{\nhat}[1]{\widehat{n}_{#1}}
\newcommand{\Chat}[1]{\widehat{C}_{#1}}
\newcommand{\expten}[1]{\cdot 10^{#1}}
%
% Formation energies
\newcommand{\dG}[1]{\Delta G_{#1}}
\newcommand{\dGs}[1]{\Delta G_{#1,s}}
\newcommand{\dGhat}[1]{\Delta\widehat{G}_{#1}}
%
% Nucleation notation
\newcommand{\Jminus}[1]{J_{#1-{\scriptscriptstyle 1/2}}}
\newcommand{\Jplus}[1]{J_{#1+{\scriptscriptstyle 1/2}}}
\newcommand{\JminusInline}[1]{J_{#1-1/2}}
\newcommand{\JplusInline}[1]{J_{#1+1/2}}
\newcommand{\timelag}[1]{\tau_{#1}}
\newcommand{\zel}{Z_{\icrit}}
%
% substances
\newcommand{\noctane}{\mbox{$n$-oc\-tane}{}}
\newcommand{\isooctane}{\mbox{iso-oc\-tane}{}}
\newcommand{\nhexane}{\mbox{$n$-hex\-ane}{}}
\newcommand{\nnonane}{\mbox{$n$-no\-nane}{}}
\newcommand{\ndecane}{\mbox{$n$-de\-cane}{}}
\newcommand{\nalkanes}{\mbox{$n$-al\-kanes}{}}
\newcommand{\nitrogen}{\mbox{N$_2$}{}}

\newtheorem{thm}{Theorem}[section]
\newtheorem{defn}[thm]{Definition}
\newtheorem{rem}[thm]{Remark}
\newtheorem{prop}[thm]{Proposition}
\newtheorem{lem}[thm]{Lemma}

\newcommand{\FronTier}{\textit{Fron\hspace{-2pt}T\hspace{-1pt}ier\hspace{2pt}}}

%From amsart.cls
%\newenvironment{pf}{\proof[\proofname]}{\endproof}
%\newenvironment{pf*}[1]{\proof[#1]}{\endproof}
 
%{\theoremstyle{plain}
%\newtheorem{thm}{Theorem}[section]
%\newtheorem{thm}{Theorem}
%\newtheorem{prop}[thm]{Proposition}
%\newtheorem{cor}[thm]{Corollary}
%\newtheorem{lem}[thm]{Lemma}
%\newtheorem{conj}[thm]{Conjecture}
%}
  
%{\theoremstyle{definition}
%\newtheorem{defn}[thm]{Definition}
%\newtheorem{assump}[thm]{Assumption}
%}
   
%{\theoremstyle{remark}
%\newtheorem{rem}{Remark}
%\newtheorem{notation}{Notation}
%\newtheorem{example}{Example}
%\newtheorem{summary}{Summary}
%}
%\renewcommand{\therem}{}
%\renewcommand{\theexample}{}
%\renewcommand{\thesummary}{}



