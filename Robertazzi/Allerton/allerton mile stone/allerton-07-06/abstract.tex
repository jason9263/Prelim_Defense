This thesis considers two problems.  One problem is closed-form solutions for equivalence computation \cite{robertazzi1993processor} of divisible workload in a mesh networks and the other problem is scheduling divisible workloads from multiple sources in mesh networks of processors.  We propose a flow matrix closed-form equation to present the equivalence, which allows a characterization of the nature of minimal time solution and a simple method to determine when and how much load to distribute for processors.  In addition, we also propose a rigorous mathematics proof about the flow matrix optimal solution existence and unique.  Also, we propose the use of a reduced Manhattan distance Voronoi diagram algorithm (RMDVDA) to minimize the overall processing time of these workloads by taking advantage of the processor equivalence technique.  The user case studies with $10$ sources of workloads are presented to illustrate the general approach for multiple sources of workloads.  In the first phase, a Voronoi Manhattan distance diagram is used to obtain a network cluster division.  In the second phase, we propose an efficient algorithm to obtain near-optimal load distribution among processors represented by equivalent processors.  The algorithm minimizes the number of processors utilized.  Experimental evaluation through simulations demonstrates that a task can be finished in the same suboptimal time and yet save about $30\%$ of processor resources.  Further, the lower band of intuitive and heuristic algorithm is also investigated.